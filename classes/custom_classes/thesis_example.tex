% Auth: Nicklas Vraa
% Docs: https://github.com/NicklasVraa/LiX

\documentclass{thesis}
\usepackage{lipsum}

\lang     {english}
\title    {An Interesting and Cool Title}
\subtitle {And An Equally Interesting Subtitle}
\authors  {Nicklas Vraa}{Second Author}
\keywords {Research, template, packages}

\begin{document}
\toc

\h*{Non-Numbered Heading}
\lipsum[1]

\h{Numbered Heading}
\lipsum[2]

\hh{Subheading}
\lipsum[3]

\hhh{Subsubheading}
\lipsum[4]

\hhhh{Subsubsubheading}
\lipsum[5-10]

\hh{Formats}
Nunc ante \b{bold} lectus, pretium id \i{italic} sodales, dapibus \s{strikethrough} urna. Suspendisse maximus \u{underlined} metus sed ante commodo efficitur. Some inline code: \c{def func(a,b): return a+b}

\hh{Mathematics}
Pellentesque sagittis orci ut lorem blandit, vel cursus urna interdum. Mauris malesuada fermentum ipsum, accumsan varius velit porttitor ut. Lorem ipsum dolor sit amet, consectetur adipiscing elit. Etiam rutrum sem orci, eget ornare justo sodales.

    \math{my_equation}{
    \hat{x}_i = \frac{x_i-\mu}{\sqrt{\sigma^2+\epsilon}}
    }

\hh{Code Blocks}
Nullam congue ligula vitae urna convallis commodo. Proin nunc mi, vulputate quis viverra eu, consequat vitae risus sed venenatis. Praesent ut libero in dui mattis maximus.

    \code{my_code}{python}{
    # Mauris viverra massa id lorem pretium gravida.

    if num == 1:
      print(num, "is not a prime.")
    elif num > 1:
      for i in range(2, num):
        if (num % i) == 0:
          print(num, "is a prime.")
          break
    }
    {This is some code.}

\hh{Tables}
Cras vitae sem egestas, elementum felis vitae, ultricies ante. Fusce pellen tesque massa vitae massa molestie, at cursus urna scelerisque. Aliquam malesuada nunc at est vulputate condimentum. Aliquam luctus tellus id ipsum facilisis facilisis. Cras eu egestas magna.

    \cols{2}{
    \tabs{my_table}{cols}{
    This & is & a & table \\
    1    & 2  & 3 & 4     \\
    a    & b  & c & d
    }{This is a cols-table. Notice that the description wraps neatly.}
    \tabs{my_table}{rows}{
    So   & 1 & 2 & 3 & 4 \\
    is   & 5 & 6 & 7 & 8 \\
    this & 9 & 0 & 0 & 0
    }{And this is a rows-table}
    }

\hh{Figures}
Nullam massa nunc, sollicitudin id eleifend vitae, pellentesque sit amet lectus. Morbi vestibulum leo quis tempor lacinia.

    \fig{my_svg}{1}{resources/placeholder.svg}
    {This is an svg-figure scaled by 1, and this is a very long description.}

Integer sed metus malesuada, volutpat urna condimentum, aliquet metus. Phasellus interdum.

    \fig{my_png}{0.7}{resources/placeholder.png}
    {This is a png-figure scaled by 0.7.}


\hh{Columns}
Lorem ipsum dolor sit amet, consectetur adipiscing elit. Cras malesuada tortor ac condimentum molestie. In hac habitasse platea dictumst.

    \cols{2}{
    \fig{left_fig}{1}{resources/placeholder.png}{This figure has a scale of 1.}
    \fig{right_fig}{1}{resources/placeholder.png}{As does this figure.}
    }

\hh{Algorithms}
Nulla odio tortor, feugiat sit amet justo quis, placerat egestas odio. Vestibulum cursus nulla lectus, id congue neque laoreet eget. Pellentesque et blandit leo.

    \algo{my_alg}{
    let $x \in \Z$
    if $x = 1$
      do stuff
    else
      do another thing
    }
    {This is an algorithm described in pseudo-code.}

\hh{Lists}
Elit vel sagittis luctus, arcu libero pellentesque nisi, sed consectetur quam neque a elit. Donec consectetur cursus nulla eu feugiat. Lorem ipsum dolor sit amet, consectetur adipiscing elit.

    \items{
    \item Something.
    \item Another thing.
        \items{
        \item A subitem.
        \item Another subitem.
            \items*{
            \item A bullet point.
            \item Another bullet point.
            }
        }
    \item And another item.
    \item Last item.
    }

\hh{Referencing}
Here is a hyperlink to a \url{webpage}{https://www.overleaf.com/}. This is a reference to \at{my_equation}, and this refers to \at{my_svg}. You can also refer to headings like \at{Tables}. \At{my_code} is a capitalized variant. For all references, both the name and number are links. Of cource, you can cite sources using \c{\cite{...}}, like this \cite{minted} or this \cite{tabularray}. You can also insert a bibliography.

\hh{Conclusion}
Aliquam luctus tellus id ipsum facilisis facilisis. Cras eu egestas magna. Aenean id nunc odio. Integer sit amet justo mi. Integer lorem neque, sodales et tellus finibus.

\bib{resources/refs}

\end{document}
