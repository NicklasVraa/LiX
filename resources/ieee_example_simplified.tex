% Auth: Nicklas Vraa
% Docs: https://github.com/NicklasVraa/LiX

\documentclass{ieee}

\size     {letter}
\header   {LiX IEEE Journal Template for \LaTeX}
\title    {A Sample Document for IEEE Journals and \\
           Transactions Using a LiX-based Class}
\authors  {Name Lastname}{Another Guy}{And Another}
\keywords {Research, templates, metapackages, packages,
           LaTeX, IEEE, LiX, Simplicity, Abstraction}
\idnum    {1234-56789}

\abstract{% ...}

\begin{document}

\h*{Non-Numbered Heading} % ...

\h{Numbered Heading} % ...

\hh{Subheading} % ...

\hhh{Subsubheading} % ...

\h{Formats}
Nunc ante \b{bold} lectus, pretium id \i{italic} sodales, dapibus \s{strikethrough} urna. Suspendisse maximus \u{underlined} metus sed ante commodo efficitur. Some inline code: \c{def func(a,b): return a+b} % ...

\h{Mathematics}
% ...
    \math{my_equation}{
        \hat{x}_i = \frac{x_i-\mu}{\sqrt{\sigma^2+\epsilon}}
    }

\h{Code Blocks}
 % ...
    \code{my_code}{python}{
    # Mauris viverra massa id lorem pretium gravida lorem ipsum.

    if num == 1:
      print(num, "is not a prime.")
    elif num > 1:
      for i in range(2, num):
        if (num % i) == 0:
          print(num, "is a prime.")
          break
    }
    {This is some code.}

\h{Tables}
% ...
    \tabs{my_table}{cols}{
    This & is & a & cool & table \\
    1    & 2  & 3 & 4    & 5     \\
    a    & b  & c & d    & e
    }
    {This is a table - Notice that the description wraps.}

\h{Figures}
 % ...
    \fig{my_svg}{0.7}{resources/placeholder.svg}
    {This is an svg-figure scaled by 0.7, and this is a very long description.}

    \fig{my_png}{0.7}{resources/placeholder.png}
    {This is a png-figure.}

\hh{Lists}
 % ...
    \items*{
        \item This is a very long item to test the wrapping of text in the unordered environment.
        \items*{
            \item Another item, but indented.
            \items*{
                \item Yet another item.
                \item An item on the same level.
            }
        }
    }
 % ...
    \items{
        \item This is an item.
        \items{
            \item Another item, but indented.
            \items{
                \item Yet another item.
                \item An item on the same level.
            }
        }
        \item Last item.
    }

 % ...

\h{Referencing}
Here is a hyperlink to a \url{webpage}{https://www.overleaf.com/}. This is a reference to \at{my_equation}, and this refers to \at{my_svg}. You can also refer to headings like \at{Tables}. \At{my_code} is a capitalized variant. For all references, both the name and number are links. Of cource, you can cite sources using \c{\cite{...}}, like this \cite{minted} or this \cite{tabularray}. You can also insert a bibliography.

\h{Conclusion}
 % ...

\toc % table-of-contents.

\bib{resources/refs}{ieeetr}

\end{document}
